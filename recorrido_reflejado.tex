\documentclass{article}
\usepackage[spanish,activeacute]{babel}
\usepackage{changepage}
\usepackage[utf8]{inputenc}

%% Datos del documento
\title{Recorridos en árboles reflejados.}
\author{Mario Román}


%% Cuerpo del documento
\begin{document}

\maketitle

\section{Árboles reflejados.}
\subsection{Definiciones}
\textbf{ Enunciado. Encontrar la relación entre los distintos recorridos en un
  árbol binario y su árbol reflejado.
} \\

Un árbol binario será para nosotros una raíz de etiqueta $a$, unida
a dos subárboles ordenados, izquierdo y derecho, $A_1$ y $A_2$.
Un árbol binario es entonces de la forma $A = (A_1,a,A_2)$ o vacío $A = nulo$.
Nótese que la división entre árboles y nodos es aquí casi inexistente,
cada nodo es en sí un árbol con sus descendientes.\\
Vamos a definir además la función que refleja al árbol y la función que invierte 
un orden dado, para ello aplicamos inducción sobre la altura del árbol, que será
siempre menor en una unidad en un subárbol frente al árbol inicial:\\

Definimos el reflejado de $A$, $rfl(A)$ como:
\begin{adjustwidth}{2cm}{}
  $\displaystyle rfl(A_1,a,A_2) = \left(rfl(A_2), a, rfl(A_1)\right)$\\
  $\displaystyle rfl(nulo) = nulo$\\
\end{adjustwidth}

Definimos el inverso de una ordenación de etiquetas, $inv([a_1,a_2,a_3, \dots, a_n])$ expresando el orden
como secuencia de etiquetas o como unión de otros órdenes:
\begin{adjustwidth}{2cm}{}
  $\displaystyle inv([a_1,a_2,\dots]) = inv([a_2,a_3,\dots]) \cup [a_1]$\\
  $\displaystyle inv(J \cup K) = inv(K) \cup inv(J)$\\
  $\displaystyle inv([\ ]) = [\ ]$\\
\end{adjustwidth}

Usaremos además las funciones postorden ($pos()$), preorden ($pre()$) e inorden ($ino()$), ya definidas,
que expresamos en esta notación como: \\
\begin{adjustwidth}{2cm}{}
  $pre(nulo) = [\ ]$\\*
  $pos(nulo) = [\ ]$\\*
  $ino(nulo) = [\ ]$\\*
  $pre(A_1,a,A_2) = [a] \cup pre(A_1) \cup pre(A_2)$\\*
  $pos(A_1,a,A_2) = pos(A_1)  \cup pos(A_2) \cup [a]$\\*
  $ino(A_1,a,A_2) = ino(A_1) \cup [a] \cup ino(A_2)$\\
\end{adjustwidth}

Nótese que en todo momento estamos usando el operador $\cup$ para unir de manera ordenada los órdenes de 
etiquetas resultado de los recorridos sobre el árbol, no como unión conjuntista.




\bigskip  
\subsection{Preorden y postorden}
\textbf{El preorden del reflejado de un árbol es el inverso de su postorden. Es decir,
  \begin{adjustwidth}{2cm}{}
   $\displaystyle pre(rfl(A)) = inv(pos(A))$
  \end{adjustwidth}
}
\medskip

\noindent Demostramos usando las definiciones por inducción sobre la altura. Para el caso del árbol
vacío es trivial:
\begin{adjustwidth}{1cm}{}
 $\displaystyle pre(rfl(nulo)) = pre(nulo) = [\ ]$\\
 $\displaystyle inv(pos(nulo)) = inv([\ ]) = [\ ]$\\
\end{adjustwidth}

\noindent Y para el caso inductivo, tomando como hipótesis de inducción que está demostrado para $A_1,A_2$.
\begin{adjustwidth}{0.5cm}{}
 \mbox{$\displaystyle pre(rfl(A_1,a,A_2)) = pre(rfl(A_2),a,rfl(A_1)) = [a] \cup pre(rfl(A_2)) \cup pre(rfl(A_1))$} \\
 \mbox{$\displaystyle inv(pos(A_1,a,A_2)) = inv(pos(A_1) \cup pos(A_2) \cup [a]) = [a] \cup inv(pos(A_2)) \cup inv(pos(A_1)$}
\end{adjustwidth}
Que por hipótesis de inducción, son iguales.\\


\bigskip
\textbf{El postorden del reflejado de un árbol es el inverso de su preorden. Es decir,
  \begin{adjustwidth}{1cm}{}
   $\displaystyle pos(rfl(A)) = inv(pre(A))$
  \end{adjustwidth}
}
\medskip

\noindent Esto es trivial sabiendo que $rfl(rfl(A)) = A$, y que $inv(inv(o)) = o$, y aplicando el resultado anterior al árbol reflejado.
\begin{adjustwidth}{1cm}{}
 $\displaystyle pre\left(rfl(rfl(A))\right) = inv\left(pos(rfl(A))\right)$\\
 $\displaystyle inv(pre(A)) = inv(inv(pos(rfl(A))))$\\
 $\displaystyle inv(pre(A)) = pos(rfl(A))$
\end{adjustwidth}


\bigskip
\subsection{Inorden}
\textbf{El inorden de un árbol es el inverso del inorden de su reflejado. Es decir,
  \begin{adjustwidth}{2cm}{}
    $ino(rfl(A)) = inv(ino(A))$
  \end{adjustwidth}
}
\medskip
\noindent Demostramos de nuevo por inducción sobre la altura. Para el caso trivial:
\begin{adjustwidth}{1cm}{}
  $ino(rfl(nulo)) = ino(nulo) = [\ ]$\\
  $inv(ino(nulo)) = inv([\ ]) = [\ ]$
\end{adjustwidth}

\noindent Y ahora, suponiendo que esté demostrado para $A_1,A_2$, se tendrá: 
\begin{adjustwidth}{1cm}{}
 \mbox{$ino(rfl(A_1,a,A_2)) = ino(rfl(A_2),a,rfl(A_1)) = ino(rfl(A_2)) \cup [a] \cup ino(rfl(A_1))$}\\
 \mbox{$inv(ino(A_1,a,A_2)) = inv(\ ino(A_1) \cup a \cup ino(A_2)\ ) = inv(ino(A_2)) \cup [a] \cup inv(ino(A_1))$}\\
\end{adjustwidth}
Donde, por hipótesis de inducción, ambos son iguales.


\end{document}